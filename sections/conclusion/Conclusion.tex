\chapter{Conclusion}
We specified a revised problem statement in \autoref{sec:problemstatement}:
\begin{quote}
    \textit{How can we design and implement a scalable, web-based solution to connect prospective learners and personal tutors, and highlight services that a user is expected to like?}
\end{quote}
In the problem analysis \autoref{ch:problem-analysis} we defined some guidelines for creating a scalable system, and we followed these guidelines when implementing the different components of the system.
We designed the system by creating a series of diagrams as well as prototypes for the user interface in \autoref{ch:design}.
We implemented our system using React for the front end and NodeJS with TypeScript for the API.
For highlighting services that a user is expected to like we used collaborative filtering to recommend services based on the user's previous ratings.
To ensure that the system was of a certain quality we used different kinds of testing.
Formal reviews, unit testing, and some integration testing was used to ensure the quality of the system.
In addition to this kind of testing we also used usability testing to ensure that the system met the needs of the users.
Based on the results described in \autoref{sec:usability-test-section}, we determine that the design does not have any major flaws.
In conclusion, as described in \autoref{sec:moscow} we achieved a minimal viable product through completing all our must have and should have requirements, and therefore fulfilled the problem statement.