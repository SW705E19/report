\section{Introduction to React}
React is a library for JavaScript that we will be using to build the frontend of the system.
In this section we will look at the different building blocks and concepts of React.
React will often be used with a syntax extension to JavaScript called JSX \cite{introducingJSX} which makes it possible to declare variables that resemble HTML code.

\begin{center}
    \texttt{const element = <h1>Hello, world!</h1>;}
\end{center}
This variable declaration can be called a React element which can then be rendered to the Document Object Model (DOM) of the web page.
With React, it is possible to render the element to the DOM by using \texttt{ReactDom.render()}.
The element is passed to this function as well as the DOM node we would like it to be rendered to.
In this case we are rendering the React element \texttt{element} to the DOM node \texttt{root}.

\begin{center}
    \texttt{ReactDOM.render(element, document.getElementById('root'));}
\end{center}

A nice feature of JSX is that it allows you to include dynamic content in HTML, by including JavaScript variables wrapped in curly braces to indicate that they should be evaluated by the JavaScript engine upon rendering.

\begin{lstlisting}
const name = 'Josh Perez';
const element = <h1>Hello, {name}</h1>;

ReactDOM.render(
  element,
  document.getElementById('root')
);
\end{lstlisting}
The variable that is embedded in the JSX can be any valid JavaScript expression \cite{introducingJSX}.
\\\\
In React we use components, which are files that hold all of the logic, HTML and styling that we need.

\begin{lstlisting}
import React from "react";

function App() {
    return (
        <div>
            <h1>Hello React</h1>
        </div>
    );
}

export default App
\end{lstlisting}
In this example we create the component \texttt{App} and export it.
By exporting it we can use the \texttt{App} component elsewhere and render it.
In the function we return a JSX element which React will then compile to regular JavaScript that can be understood by the browser.
The idea is to build the UI from components by exporting them and then importing them into other larger components to contain them.
This means that the components act as the building blocks of the UI because we can reuse them in other components.
\begin{lstlisting}
import React from 'react';

function Button() {
    return (
        <div>
            <button onClick={console.log('Hello')}>Hello React</button>
        </div>
    );
}

export default Button
\end{lstlisting}
As an example we can create a button component with an on-click event embedded into it.
Here we use brackets to enable us to embed JavaScript in the JSX which will be run when the button is clicked.
We can now export this button and it can be imported by other components.
A component is conceptually a JavaScript function, which means that they can accept inputs called \texttt{props} which can contain data that can be used in the React elements.
If a component is used in another component, we can use \texttt{props} to pass data to the child component.
\\\\
In general, \texttt{props} should be read-only, which means a component should not try to change its values.
To allow the UI elements to change over time we instead make use of \textit{State}, which allows the components to change their output over time when handling user inputs or network responses.
\textit{State} is similar to \texttt{props} but is instead private and only the specific component can make changes to it.
To make use of \textit{State} we need to initialize the component as a class instead of a function.
\begin{lstlisting}
    function Button() {
        return (
            <div>
                <button onClick={console.log('Hello')}>Hello React</button>
            </div>
        );
    }
\end{lstlisting}
We can convert this into a class like this:
\begin{lstlisting}
    class Button extends React.Component {
        render() {
            return (
                <div>
                    <button onClick={console.log('Hello')}>Hello React</button>
                </div>
            );
        }
    }
\end{lstlisting}
This is useful since the render method is called every time the local \textit{State} is changed.
We can set the initial state of a component by assigning to \texttt{this.state}
\begin{center}
    \texttt{this.state = \{date = new Date();\};}
\end{center}
The state can then be changed by reassigning to the \texttt{date} variable.
This is done by using the \texttt{this.setState()} function.
\begin{center}
    \texttt{this.setState(\{date = new Date();\});}
\end{center}
When a component has a state set it is considered to be a stateful component.
Parent or child components should not care if a certain component is stateful or stateless since the state is local and not accessible by other components.
A component can however pass its state through \texttt{props} to a child component, which means that information about a state can only be passed in a \textit{top-down} approach \cite{ReactJS}.
