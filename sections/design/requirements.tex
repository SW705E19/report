\section{Requirements}
In this section, the functional and general requirements for this project will be presented.
This serves as a way to ensure that the project delivered can be qualified as a minimum viable product.

\subsubsection{Functional requirements}
The functional requirements define some code related requirements. These requirements should be presented in a binary manner, such that it is trivial to evaluate whether the requirement has been fulfilled or not.

\begin{itemize}
    \item Users being able to contact tutors about courses that the tutor provides
    \item Users being able to review tutors
    \item Users and tutors being able to interact with each other through a messaging system
    \item Tutors being able to upload and share material with users
    \item Users being able to pay tutors for their courses
    \item Users needs to be able to find information on:
    \begin{itemize}
        \item Name
        \item Price
        \item Location
        \item Contact information
        \item Education
    \end{itemize}
\end{itemize}

\subsubsection{Non functional requirements}
These non-functional requirements are more difficult to evaluate than the functional requirements.
The reasoning behind this is that these requirements may need to be evaluated through a usability test, or that the criteria cannot be evaluated until a later point. 

\begin{itemize}
    \item Usability:
    \begin{itemize}
        \item It needs to be intuitive
        \item It needs to be easily learnable for new users
    \end{itemize}
    \item Maintainable:
    \begin{itemize}
        \item It needs to be tested
        \item Make future functionality easy to integrate into the existing codebase
    \end{itemize}
    \item Reliability:
    \begin{itemize}
        \item If an error happens, the user should be notified of the error in a way that they can understand
    \end{itemize}
    \item Scalability:
    \begin{itemize}
        \item It needs to be possible to add additional users to the system
        \item It needs to be easy to add additional language
    \end{itemize}
\end{itemize}
\noindent
In the next section, we will go into further details with the requirements, and how the requirements are prioritized to form the minimum viable product.
