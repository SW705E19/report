\section{Why React?}
For our frontend, we considered two frameworks and a library: Vue.js, Angular and React.
Each of these frameworks and libraries are very popular web development tools and are used by many different companies\cite{VueReactAngular}.
\\
Each of these have extended documentation and are maintained by reputable companies.
\\\\
Angular is the oldest of the three options we looked into. 
Angular is created and maintained by Google.
It is generally considered the option with the steepest learning curve.
Performance wise Angular should be the slowest of the three and also the one that takes up the most space.
This is because Angular also provides templates and testing utilities for its user.
It is a complete package and should provide the developer with all the functionality he needs for a frontend project.
\\\\
Vue is the newest of the three. 
It was developed by an ex-engineer of Google and is the one that uses the least amount of space, making it good for small web projects.
Vue is considered to be the easiest of the three to learn.
Vue, like Angular, is also a framework and is designed around its ecosystem which offers its developers add ons like Vue Router which makes it a bit more flexible than Angular since it does not come in one big package.
\\\\
React, unlike the two other, is only a library, not a lot is provided in the React library which in turn gives the developer the option to choose what programming tools he or she wants to use in their project.
It is developed and maintained by Facebook.
Space wise it only uses a bit more than Vue which still makes it a good choice for small projects.
\\\\
Most of the team already used Angular for a project beforehand and wanted to try one of the other two web tools, ruling out Angular. 
\\
The decision between Vue and React was difficult but the extra flexibility of react spiked the most interest from the team even though Vue should be easier to learn according to the internet.

