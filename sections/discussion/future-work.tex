\section{Future work}
The tasks for future work focuses on completing the tasks in our MoSCoW.
There are nine features that we would like to implement.

\begin{itemize}
    \item Update the select multiple dropdown
    \item Make tutor info inactive
    \item Applying to become a tutor
    \item Upload and view material
    \item Messaging system
    \item Calendar system
    \item On-site payment
    \item Booking a tutor
    \item Extend the recommender to have both a content based and collaborative approach.
\end{itemize}
\noindent
\textit{Update the select multiple dropdowns}:
During \autoref{sec:usability-test-section} we found that our current style of selecting multiple options was unintuitive. 
They did not seem to understand that they were able to select multiple languages, as it only highlighted the selected items with a grey color.
These select multiple dropdowns should be edited, so that it becomes more intuitive, or a more clear description should be written prior to selecting, properly indicating that it is possible to select multiple options.
\\\\
\textit{Make tutor info inactive}:
If a tutor gets their tutor role removed, they will still appear on the show services page. 
This is because we do not delete tutor information or their created services.
This could be fixed by adding an additional attribute to the tutor info table in the database to see if they are active.
If active is set to false, we should not fetch the tutor info or services.
\\\\
\textit{Applying to become a tutor:}
Currently it is not possible to apply to become a tutor.
This should of course be changed, such that users have the opportunity to apply for that. 
\\\\
\textit{Upload and view material}:
We would like the option for tutors to upload material for users. 
This would make it easier for tutors to share assignments and videos that they have made. 
This was not absolutely essential for the minimum viable product, but would make the user experience for tutors and users better.
\\\\
\textit{Messaging system}: 
It would be nice with an on-site messaging system, so that tutors and users do not have to share personal information with other users of the site.
The users and tutors are still able to get in contact with one another with their contact information, but there is the possibility that their information can be misused.
\\\\
\textit{Calendar system}:
The calendar system would improve the user experience of the system. 
This would make it easier for users to see when certain tutors are available and they would be able to contact the ones who suits their schedule best.
\\\\
\textit{Booking a tutor}:
Booking a tutor is closely related to the calendar system.
The option to book a tutor should be through the tutor's calendar. 
The user should be able to request a meeting and then the tutor should have the option to either accept or decline the request.
If the meeting is accepted the tutor's and user's time slot in the calendar should be registered as occupied, so that other users will not be able to book the tutor in the same time slot.
Users should also not be able to book another tutor in their occupied time slot.
\\\\
\textit{On-site payment}:
On-site payment is one of the less important future work tasks, as tutors and users easily can exchange money with other methods.
It would however still be a nice feature for them to have the option for on-site payment.
\\\\
\textit{Extend the recommender}:
We would like to improve the recommender system, so that it has a content based and collaborative approach, as these could complement each other as described in \autoref{sec:recommender-pros-and-cons}.
Our system can benefit heavily from combining both approaches.
Currently, the system has issues with cold starts since the collaborative approach struggles with new services, as it needs ratings to create predictions.
Adding a content-based approach in which ratings could be predicted by analyzing similar services could mitigate this.
A complementary approach could also lessen the impact of the strong assumptions associated with collaborative filtering, mainly that a user's taste will not change over time.
