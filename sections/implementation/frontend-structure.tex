\section{Front end structure}
This chapter focuses on the implementation of the front end in React.
Some basic features within React were previously described in \autoref{sec:intro-to-react}.

\subsection{States}
For many of our pages we use states to store a component's dynamic data. 
An example of this is the edit service page. 
The constructor for this page is seen on \autoref{EditService-state}.
The constructor takes a variable called props as an argument, props contains all of the arguments that the component received when it was instantiated. 
Within the constructor \texttt{super(props)} is called.
This needs to be done to be able to use \texttt{this.props} to access the props anywhere in the component.
Dynamic variables that will be used on the page are initialized in \texttt{this.state}. 
However, if another variable is needed, it can be added to the state at any time as the state is dynamic.\\\\
Variables in the state can be changed by calling the react hook \texttt{this.setState()} and this will automatically trigger a new rendering of the component and all of its children.
For example, it is easy to validate text fields and send an error message if the text input is invalid.  
After the state declarations, we bind the functions to the component instance using the \texttt{.bind(this)} function.
This has to be done if the function uses \texttt{this.setState} to change the state of the component.
Generally it needs to be done for all functions that need to access React hooks through \texttt{this}.
\begin{lstlisting}[caption={Constructor and state for edit service}, captionpos=b, label={EditService-state}]
class EditService extends React.Component {
    constructor(props) {
        super(props);
    
        this.state = {
            service: null,
            categories: [],
            chosenCategoryNames: [],
            redirectService: false,
            redirectOwnUser: false,
            openAlert: false
        };
            
        this.handleOnChange = this.handleOnChange.bind(this);
        this.handleOnChangeCategories = this.handleOnChangeCategories.bind(this);
        this.submit = this.submit.bind(this);
        this.deleteService = this.deleteService.bind(this);
        this.handleClickCancel = this.handleClickCancel.bind(this);
        this.handleClickDelete = this.handleClickDelete.bind(this);
        this.handleClickOpen = this.handleClickOpen.bind(this);
    }
    (*@\centerline{\raisebox{-1pt}[0pt][0pt]{$\vdots$}}@*)
\end{lstlisting}

\subsection{Services}
Services are used to send or retrieve data from the API. 
\autoref{get-by-id} shows an example or retrieving a user by their id from the database.
The \texttt{requestOptions} used as a parameter defines the type of method you use.\\
For \texttt{'POST'} methods the \texttt{requestOptions} will contain the data to send to the API as the body of the request.
For many requests it is necessary for the user to be logged in, which is defined in the header of the request through the function \texttt{authHeader()}, which checks if the user is logged in.\\
The fetch is then called with the URL link and the response is handled with the \texttt{handleResponse} function. 
This is a function that checks for errors in the response, and ensures that the user requesting the response has the proper privileges to access it.

\begin{lstlisting}[caption={Function to get a user by ID.}, captionpos=b, label={get-by-id}]
function getById(id) {
	const requestOptions = { method: 'GET', headers: authHeader() };
	return fetch(
        `http://${process.env.REACT_APP_API_URI}:
           ${process.env.REACT_APP_API_PORT}/api/users/${id}`,
		requestOptions
	)
		.then(handleResponse)
		.then(data => {
			return JSON.parse(data);
		});
}
\end{lstlisting}

\subsection{Components}
There are two types of components. 
The first one is a stateful component, also called a container.
These are typically class components and have a state.
They often contain some kind of logic and some conditional rendering dependent on the current state.
The second is a presentational component.
These are most often stateless and are called presentational since all they should do is output UI elements \cite{Vumbula-react}.

\begin{figure}[H]
    \centering
    \includegraphics[scale=0.5]{figures/create-user-error.png}
    \caption{Error message when creating a user}
    \label{fig:create-user-error}
\end{figure}
\noindent
The container for the create user page can be seen on \autoref{create-user}.
The \texttt{CreateUser} class only contains the state, some functions and a \texttt{render()}.
Within the state object of \texttt{CreateUser}, there is an object called \texttt{firstName}, which contains a \texttt{firstName} string and a \texttt{firstNameValid} boolean to indicate if \texttt{firstName} is valid.
It is initialized as true, so that the error messages are not visible when entering the page.
When they press the submit button, the \texttt{handleSubmit} function will be called, which checks if the values are valid.
If the values are invalid, the booleans in the state will be updated. 
If these are set to false, an error message will be shown as can be seen in \autoref{fig:create-user-error}.

The presentational component \texttt{UserForm} is rendered in the \texttt{render()} function. 
\begin{lstlisting}[caption={Component to create user}, captionpos=b, label={create-user}]
export class CreateUser extends Component {
    constructor(props) {
        super(props);
        this.state = {
            firstName: { firstName: '', firstNameValid: true },
            lastName: { lastName: '', lastNameValid: true },
            (*@\centerline{\raisebox{-1pt}[0pt][0pt]{$\vdots$}}@*)
        };
        this.handleChange = this.handleChange.bind(this);
        this.handleSubmit = this.handleSubmit.bind(this);
    }
    (*@\centerline{\raisebox{-1pt}[0pt][0pt]{$\vdots$}}@*)
    render() {
        return (
            <UserForm
                firstName={this.state.firstName}
                lastName={this.state.lastName}
                (*@\centerline{\raisebox{-1pt}[0pt][0pt]{$\vdots$}}@*)
            />
        );
    }
\end{lstlisting}
The implementation of \texttt{UserForm} can be seen on \autoref{user-form}.
This presentational component is stateless and uses the props that were passed to the \texttt{UserForm} component when it was instantiated in the \autoref{create-user} container.
Within the \texttt{TextField} it checks through props if the first name is valid. 
If it is not valid an error will be shown, and the \texttt{helperText} will also be shown.
Whenever the \texttt{TextField} is changed the \texttt{handleChange} is called, and as it changes the state of the parent component the page will be re-rendered.

\begin{lstlisting}[caption={Presentational component for userform}, captionpos=b, label={user-form}]
function UserForm(props) {
    const { t, classes } = props;
    
    return (
        <Container maxWidth="sm">
            <div className={classes.paper}>
                <Typography align="center" variant="h4">
                    {t('registerasauser')}
                </Typography>
                <div className={classes.form}>
                <Grid container spacing={2} direction="row">
                    <Grid item xs={12} sm={6}>
                        <TextField
                            name="firstname"
                            label={t('firstname')}
                            onChange={props.handleChange}
                            error={!props.firstName.firstNameValid}
                            helperText={
                                props.firstName.firstNameValid ? '' : t('typefirstname')
                            }
                        />
                        (*@\centerline{\raisebox{-1pt}[0pt][0pt]{$\vdots$}}@*)
\end{lstlisting}

\subsection{Material-UI}
We have chosen to use the framework \textit{Material-UI}, to ease the implementation of design, and ensure it is of a certain quality.
Material-UI is a collection of React components that are easy to integrate into the project.
If we had to implement everything ourselves we would have to invest a lot of more time into making a good design.
By using Material-UI for the design we also get consistently designed pages.
Some of the most commonly used components in our application are \texttt{TextField}, \texttt{Typography}, \texttt{Grid}, \texttt{Button} or \texttt{MenuItem}.
An example of using Material-UI in the header can be seen on \autoref{material-ui}.
As seen in the begin of the file, we import the different components that are needed.
From line 11 onwards we then use them to create the design.
The header can be seen on \autoref{fig:header}.
In \autoref{material-ui} we implement parts of the header, more specifically the log in icon in the right side.
If the button is pressed, the user is redirected to the login page.
\begin{lstlisting}[caption={Use of material-ui in the header}, captionpos=b, label={material-ui}]
import {
    List,
    ListItem,
    ListItemIcon,
    ListItemText
} from '@material-ui/core';
import {
	ExitToApp
} from '@material-ui/icons';
(*@\centerline{\raisebox{-1pt}[0pt][0pt]{$\vdots$}}@*)
<List>
    <ListItem button component={Link} to="/login" key="LoginCircle" name="login">
        <ListItemText primary="Login" />
        <ListItemIcon>
            <ExitToApp fontSize="large" />
        </ListItemIcon>
    </ListItem>
(*@\centerline{\raisebox{-1pt}[0pt][0pt]{$\vdots$}}@*)
\end{lstlisting}


\begin{figure}[H]
    \includegraphics[width=\linewidth]{figures/header.png}
    \caption{The header for the application}
    \label{fig:header}
\end{figure}
