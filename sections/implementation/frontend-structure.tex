\section{Frontend structure}
This chapter focuses on the implementation of the frontend in \texttt{React}.
Some basic features within \texttt{React} were previously described in \autoref{sec:intro-to-react}.


\subsection{States}
For many of our pages we use states to store a component's dynamic data. 
An example of this is the edit service page. 
The constructor for this page is seen on \autoref{EditService-state}.
Within the constructor \texttt{super(props)} is called. 
This is used to call the \texttt{React} parent constructor and pass its properties as an argument.
This is necessary whenever you want to call \textit{this} within the constructor.
Variables that will be used on the page are found within \texttt{this.state}. 
However, if another variable is needed, it can be added to the state at any time as the state is dynamic.
The page will get rendered again whenever the state changes. 
This is very useful for many things. 
For example, it is easy to validate text fields and send an error message if the text input is invalid.  
After the state declarations, we bind the functions to the component instance using the \texttt{.bind(this)} function.
This is useful as we later can call the functions in the renderer using \texttt{this}.
These functions often either change the state of the component or call the API to upload data.
\begin{lstlisting}[caption={Constructor and state for edit service}, captionpos=b, label={EditService-state}]
class EditService extends React.Component {
    constructor(props) {
        super(props);
    
        this.state = {
            service: null,
            categories: [],
            chosenCategoryNames: [],
            redirectService: false,
            redirectOwnUser: false,
            openAlert: false
        };
            
        this.handleOnChange = this.handleOnChange.bind(this);
        this.handleOnChangeCategories = this.handleOnChangeCategories.bind(this);
        this.submit = this.submit.bind(this);
        this.deleteService = this.deleteService.bind(this);
        this.handleClickCancel = this.handleClickCancel.bind(this);
        this.handleClickDelete = this.handleClickDelete.bind(this);
        this.handleClickOpen = this.handleClickOpen.bind(this);
    }
    ...
\end{lstlisting}

\subsection{Services}
Services are used to send or retrieve data from the API. 
The \texttt{requestOptions} used as a parameter defines the type of method you use.
The methods that can be used are either \texttt{'GET'}, \texttt{'POST'}, \texttt{'PUT'}, \texttt{'PATCH'} or \texttt{'DELETE'}.
\texttt{'GET'} returns what is requested if it exists.
\texttt{'POST'} adds an additional row to the database.
\texttt{'PUT'} and \texttt{'PATCH'} updates an element. 
\texttt{'PUT'} should only be used if resource is replaced, where \texttt{PATCH is if you're updating an existing resource.}
For \texttt{post} methods the \texttt{requestOptions} will contain the data to send to the API as the body of the request.
For many requests it is necessary for the user to be logged in, which is defined in the header of the request through the function \texttt{authHeader()}, which checks if the user is logged in.

The fetch is then called with the URL link and the response is handled with the \texttt{handleResponse} function. 
This is a function that checks for errors in the response, and ensures that the user requesting the response has the proper privileges to access it.

\begin{lstlisting}[caption={Function to get a user by ID.}, captionpos=b, label={material-ui}]
function getById(id) {
	const requestOptions = { method: 'GET', headers: authHeader() };
	return fetch(
		`http://${process.env.REACT_APP_API_URI}:${process.env.REACT_APP_API_PORT}/api/users/${id}`,
		requestOptions
	)
		.then(handleResponse)
		.then(data => {
			return JSON.parse(data);
		});
}
\end{lstlisting}


\subsection{Components and containers}
%Something about stateful and stateless components and how we have split it in our folder structure to reflect it.


\subsection{Material-UI}
We have chosen to use the framework \texttt{Material-UI}, to ease the implementation of design, and ensure it is of a certain quality.
\texttt{Material-UI} is a collection of \texttt{React} components that are easy to integrate into the project.
If we had to implement everything ourselves we would have to invest a lot of more time into making a good design.
By using \texttt{Material-UI} for the design we also get consistently designed pages.
Some of the most commonly used components in our application are \texttt{TextField}, \texttt{Typography}, \texttt{Grid}, \texttt{Button} or \texttt{MenuItem}.
An example of using \texttt{Material-UI} in the header can be seen on \autoref{material-ui}.
As seen in the begin of the file, we import the different components that are needed.
From line 11 onwards we then use them to create the design.
We create a list, add an item to it which is a link to the landing page, and finally give it some text and an icon.
\begin{lstlisting}[caption={Use of material-ui in the header}, captionpos=b, label={material-ui}]
import {
    List,
    ListItem,
    ListItemIcon,
    ListItemText
} from '@material-ui/core';
import {
	Home
} from '@material-ui/icons';
...
<List>
    <ListItem button component={Link} to="/" key="Home" name="home">
        <ListItemText primary="Home" />
        <ListItemIcon>
            <Home fontSize="large" />
        </ListItemIcon>
    </ListItem>
...
\end{lstlisting}

