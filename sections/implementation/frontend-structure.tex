\section{Frontend structure}
This chapter focuses on the implementation of the frontend in \textit{React}.
Some basic features within \textit{React} was previously described in \autoref{sec:intro-to-react}.


\subsection{States}
For many of our pages we use states to store a components dynamic data. 
An example of this is the edit service page. 
The constructor for this page is seen on \autoref{EditService-state}.
Within the constructor \texttt{super(props)} is called. 
This is used to call the \textit{React} parent constructor and pass props as an argument.
This is necessary whenever you want to call \textit{this.} within the constructor.
Within \texttt{this.state} is the variables that will be used on the page. 
However, as the state is dynamic, if another variable is needed, it can be added anytime to the state.
Whenever the state changes, the page will get rendered again. 
This is very useful for many things. 
For example it is easy to validate text fields and give an error message if the inputted text is invalid.  

After the state declarations, we bind the functions to the component instance using the \texttt{.bind(this)} function.
This is useful as we later can call the functions in the renderer using \texttt{this}.
These functions often either change the state of the component or call the API to upload data.
\begin{lstlisting}[caption={Constructor and state for edit service}, captionpos=b, label={EditService-state}]
class EditService extends React.Component {
    constructor(props) {
        super(props);
    
        this.state = {
            service: null,
            categories: [],
            chosenCategoryNames: [],
            redirectService: false,
            redirectOwnUser: false,
            openAlert: false
        };
            
        this.handleOnChange = this.handleOnChange.bind(this);
        this.handleOnChangeCategories = this.handleOnChangeCategories.bind(this);
        this.submit = this.submit.bind(this);
        this.deleteService = this.deleteService.bind(this);
        this.handleClickCancel = this.handleClickCancel.bind(this);
        this.handleClickDelete = this.handleClickDelete.bind(this);
        this.handleClickOpen = this.handleClickOpen.bind(this);
    }
    ...
\end{lstlisting}

\subsection{Services}
Services are used to send or retrieve data from the API. 
The \texttt{requestOptions} define the method you use, and the token that the user has.
The function \texttt{authHeader()} check if you're logged in. 
If you're logged in, a token is returned.
The method that can be used are either \texttt{'GET'}, \texttt{'POST'}, \texttt{'PUT'}, \texttt{'PATCH'} or \texttt{'DELETE'}.
\texttt{'GET'} returns what is requested if it exists.
\texttt{'POST'} adds an additional row to the database.
\texttt{'PUT'} and \texttt{'PATCH'} updates an element. 
\texttt{'PUT'} should only be used if resource is replaced, where \texttt{PATCH is if you're updating an existing resource.}

Then the fetch is called with the URL link and then the response is handled with \texttt{handleResponse}. 
This is a function that checks for errors in the response.

\begin{lstlisting}[caption={Function to get a user by ID.}, captionpos=b, label={material-ui}]
function getById(id) {
	const requestOptions = { method: 'GET', headers: authHeader() };
	return fetch(
		`http://${process.env.REACT_APP_API_URI}:${process.env.REACT_APP_API_PORT}/api/users/${id}`,
		requestOptions
	)
		.then(handleResponse)
		.then(data => {
			return JSON.parse(data);
		});
}
\end{lstlisting}


\subsection{Components and containers}
%Something about stateful and stateless components and how we have split it in our folder structure to reflect it.


\subsection{Material-UI}
We have chosen to use the framework Material-UI, to make the implementation of good design easier????
Material-UI is a collection of React components that are easy to integrate into the project.
If we had to implement everything ourselves we would have to invest a lot of more time into making a good design.
Using Material-UI design we also get the same designed things on all pages. ????
Some of the most commonly used components in our application is \texttt{TextField}, \texttt{Typography}, \texttt{Grid}, \texttt{Button}, \texttt{MenuItem} or a logo component.
A use of Material-UI in the header can be seen on \autoref{material-ui}. 
\begin{lstlisting}[caption={Use of material-ui in the header}, captionpos=b, label={material-ui}]
import {
    List,
    ListItem,
    ListItemIcon,
    ListItemText
} from '@material-ui/core';
import {
	Home
} from '@material-ui/icons';
...
<List>
    <ListItem button component={Link} to="/" key="Home" name="home">
        <ListItemText primary="Home" />
        <ListItemIcon>
            <Home fontSize="large" />
        </ListItemIcon>
    </ListItem>
...
\end{lstlisting}

