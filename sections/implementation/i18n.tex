\section{Translations}
To have a scalable application it is necessary to easily be able to translate the text to different languages.
For this purpose we have used i18next, which is an internationalization framework for JavaScript.
It is supported for many different frameworks like React, React Native, .net, Angular and many more \cite{react-i18next}.
Instead of hardcoding the text on every page, we wrap the export of the component in the function withTranslations which injects the function t into the props of the component.  

\begin{lstlisting}[caption={Translated header when registering as a user.}, captionpos=b, label={withTranslationsUserForm}]
    import { withTranslation } from "react-i18next";

    function UserForm(props){
        const { t, classes } = props;

        return (
            <Container maxWidth="sm">
               <div className={classes.paper}>
                  <Typography align="center" variant="h4">
                      {t('registerasauser')}
                  </Typography>
               ...
        );
    }

    export default withTranslation()(withStyles(styles)(UserForm))
\end{lstlisting}
\noindent
\autoref{withTranslationsUserForm} shows the translation for the header on the register page.
The \texttt{withTranslation()} function injects the function t into the props of the component. When called the function t will look up the passed key in the translation files to find the correct translation for "registerasauser". 
A screenshot of the files can be seen on \autoref{fig:translationfiles}.
\begin{figure}
    \centering
    \includegraphics[scale=0.5]{figures/translations.PNG}
    \caption{The Danish and English translation files}
    \label{fig:translationfiles}
\end{figure}
\noindent
To add an additional language a new translation file needs to be created. All the lines that are present in the other files then need to be translated to the desired language. 
It is not necessary to refactor any code to add additional languages.
