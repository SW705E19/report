\section{Recommender implementation}
As defined in \autoref{subsec:recommender-our-choice} we chose to implement a recommendation system based on collaborative filtering through matrix factorization.
In \autoref{lst:recomstart} we initially make use of our other services to get the required data to predict ratings.
To perform the calculations we need all users, all services and all ratings currently in the database.
We then initialize the necessary matrices.
These are matrices with dimensions of respectively $users \times services$, $users \times factors$, $factors \times services$. 
\begin{lstlisting}[caption={The start of the recommender}, captionpos=b, label={lst:recomstart}]
export default class Recommender {
    static async calculateRecommendations(): Promise<number[][]> {
        const allUsers = await UserService.getAll();
        const allServices = await ServiceService.getAll();
        const allRatings = await RatingService.getAll();
    
        let userServiceMatrix: number[][] = [];
        let predictedRatings: number[][] = [];
        let userFactorMatrix: number[][] = [];
        let factorServiceMatrix: number[][] = [];

        userServiceMatrix = initUserServiceMatrix(numberOfRows, numberOfCols, allUsers, allServices, userServiceMatrix);

		userServiceMatrix = populateUserServiceMatrix(allRatings, numberOfRows, numberOfCols, userServiceMatrix);
(*@\centerline{\raisebox{-1pt}[0pt][0pt]{$\vdots$}}@*)
\end{lstlisting}
When we initialize both the matrix for users and factors as well as the matrix for factors and services we initialize all entries with random values between 0 and 1.
Initialization is illustrated in \autoref{lst:initUserFactor}.
This is done to ensure we can calculate a predicted rating.
The actual value is arbitrary, but a small value is chosen, as this should make the algorithm arrive at a model faster, compared to initializing with a greater value \cite{FunkMatrixFactorization}.
\begin{lstlisting}[caption={Initializing the user and factor matrix}, captionpos=b, label={lst:initUserFactor}]
function initUserFactorMatrix(numberOfRows: number, numberOfFactors: number, userFactorMatrix: number[][]): number[][] {
    for (let i = 0; i < numberOfRows; i++) {
        userFactorMatrix[i] = []; // Initialize inner array
        for (let j = 0; j < numberOfFactors; j++) {
            userFactorMatrix[i][j] = Math.random();
        }
    }   
    return userFactorMatrix;
}
\end{lstlisting}
\autoref{lst:initUserService} illustrates how we initialize the matrix for users and services.
On the first row we insert the ids of all services, and on the first column we insert all user ids.
This is done to ensure we can get the proper ids for the users and services to recommend.
Then we insert 0 in all entries, and call another function to populate the entries in which there are ratings.
If there is a rating from a user on a given service, we find the entry corresponding to these ids and insert the rating value.
This means we end with a matrix consisting of a row of service ids, a column of user ids, the rating values and 0 for those entries without ratings.
\begin{lstlisting}[caption={Initializing the user and service matrix}, captionpos=b, label={lst:initUserService}]
    export function initUserServiceMatrix(
	numberOfRows: number,
	numberOfCols: number,
	allUsers: User[],
	allServices: Service[],
	userServiceMatrix: number[][],
): number[][] {
	userServiceMatrix[0] = [];
	for (let i = 0; i < numberOfRows; i++) {
		userServiceMatrix[i + 1] = [];
		userServiceMatrix[i + 1][0] = allUsers[i].id;
	}

	for (let i = 0; i < numberOfCols; i++) {
		userServiceMatrix[0][i + 1] = allServices[i].id;
	}

	for (let i = 1; i < numberOfRows + 1; i++) {
		for (let j = 1; j < numberOfCols + 1; j++) {
			userServiceMatrix[i][j] = 0;
		}
	}

	return userServiceMatrix;
}
\end{lstlisting}
After initializing the matrices, we start the first calculation of predictions.
To do so, we call the function in \autoref{lst:dotmatrices} to calculate the dot products of the user by factor and factor by service matrices.
It iterates over a row from one matrix and a column for the other, multiplying the corresponding entries of each matrix, and adding them together.
The sum is then inserted into the predictedRatings matrix.
\begin{lstlisting}[caption={Calculating the dot predict of a matrix}, captionpos=b, label={lst:dotmatrices}]
function dotMatrices(
    numberOfRows: number,
    numberOfCols: number,
    numberOfFactors: number,
    userFactorMatrix: number[][],
    serviceFactorMatrix: number[][],
): number[][] {
    const predictedRatings: number[][] = [];
    let sum = 0;
    for (let i = 0; i < numberOfRows; i++) {
        predictedRatings[i] = [];
        for (let j = 0; j < numberOfCols; j++) {
            for (let k = 0; k < numberOfFactors; k++) {
                sum = sum + userFactorMatrix[i][k] * serviceFactorMatrix[k][j];
            }
            predictedRatings[i][j] = sum;
            sum = 0;
        }
    } 
    return predictedRatings;
}
\end{lstlisting}
After calculating the predicted ratings, we need to determine the error and update the values of the factor matrices.
We do this with the code snippet in \autoref{lst:errorandupdates}.
To calculate the error, we use the equation defined in \autoref{fig:calculatingerror}, which simply subtracts the predicted value from the actual value.
We then use the updating rules defined in the equations in \autoref{fig:updatingq} and \autoref{fig:updatingp} to update the values of the matrices.
We define the $\gamma$ value from the equation as the \texttt{alphaValue}, the $\lambda$ value as \texttt{betaValue}, and assign them to be \texttt{0.001}, based on testing and recommended values.

\begin{lstlisting}[caption={Calculating error and updating values}, captionpos=b, label={lst:errorandupdates}]
const alphaValue = 0.001;
const betaValue = 0.001;
for (let i = 0; i < numberOfRows; i++) {
    for (let j = 0; j < numberOfCols; j++) {
        if (userServiceMatrix[i + 1][j + 1] > 0) {
            const error = userServiceMatrix[i + 1][j + 1] - predictedRatings[i][j];
            for (let k = 0; k < numberOfFactors; k++) {
                userFactorMatrix[i][k] =
                    userFactorMatrix[i][k] +
                    alphaValue * (error * serviceFactorMatrix[k][j] - betaValue * userFactorMatrix[i][k]);
                serviceFactorMatrix[k][j] =
                    serviceFactorMatrix[k][j] +
                    alphaValue * (error * userFactorMatrix[i][k] - betaValue * serviceFactorMatrix[k][j]);
            }
        }
    }
}
(*@\centerline{\raisebox{-1pt}[0pt][0pt]{$\vdots$}}@*)
\end{lstlisting}
After updating the values, we calculate the dot product of the two factor matrices again, and use the new predictions to calculate the squared error through the code in \autoref{lst:squarederror}.
The squared error is calculated with the equation defined in \autoref{fig:minimizesquarederror}.
We initially look at the matrix containing our actual ratings.
We traverse it, and when the value of an entry is greater than 0 it means that there is a rating.
We subtract the prediction from the actual rating and square it, and then for the number of factors we add the regularization by squaring the values from both factor matrices, and multiplying it with the $\lambda$ value.
\begin{lstlisting}[caption={Calculating the overall squared error}, captionpos=b, label={lst:squarederror}]
let squareError = 0;
for (let i = 0; i < numberOfRows; i++) {
    for (let j = 0; j < numberOfCols; j++) {
        if (userServiceMatrix[i + 1][j + 1] > 0) {
            squareError =
                squareError + Math.pow(userServiceMatrix[i + 1][j + 1] - predictedRatings[i][j], 2);
            for (let k = 0; k < numberOfFactors; k++) {
                squareError =
                    squareError +
                    betaValue *
                    (Math.pow(userFactorMatrix[i][k], 2) + Math.pow(serviceFactorMatrix[k][j], 2));
            }
        }
    }
}
(*@\centerline{\raisebox{-1pt}[0pt][0pt]{$\vdots$}}@*)
\end{lstlisting}
Once the algorithm completes, we need to make our data persist.
To do this, we keep a ratings table in the database consisting of predicted ratings.
To send the calculated ratings, we add them to the $user \times service$ matrix with the code in \autoref{lst:predictratingsmatrix}.
Since we want to avoid possibly recommending a service the user has already rated, whenever we encounter an entry that is not 0, we remove that rating and assign it to be 0 instead.
This means that, even if a user rated a service with the maximum value, they will not get it as a recommendation since it will now rank in the bottom with a rating of 0.
The matrix is then ready to be inserted into the database.
\begin{lstlisting}[caption={Adding the predictions to the ratings matrix}, captionpos=b, label={lst:predictratingsmatrix}]
for (let i = 0; i < numberOfRows; i++) {
    for (let j = 0; j < numberOfCols; j++) {
        if (userServiceMatrix[i + 1][j + 1] == 0) {
            userServiceMatrix[i + 1][j + 1] = predictedRatings[i][j];
        } else {
            userServiceMatrix[i + 1][j + 1] = 0;
        }
    }
}
(*@\centerline{\raisebox{-1pt}[0pt][0pt]{$\vdots$}}@*)
\end{lstlisting}
