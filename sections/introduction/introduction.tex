\chapter{Introduction}
Learning a new skill is a way to break the monotony of the average day, and presents a challenge to the student.
Many prospective learners might be intimidated by embarking on the learning process by themselves, and feel reluctant to start.
One way to alleviate this is to introduce a tutor to the student, who can help them attain knowledge related to the topic and guide them in the proper direction.
This guidance could take place in person, or through various internet options after a connection has been established between the two.
\\\\
This project is carried out on the seventh semester of the software education at Aalborg University. 
As a requirement, this project focuses on the theme, \textit{Internet}.
Because of this focus, this project will aim to create a way to connect prospective learners with individuals who are capable and interested in teaching.
This connection will be created through a web-based solution, and ensure that a connection can be made without mediations from other parties.
The solution should also focus on scalability. 
The ability to have many tutors offer their services is essential, as is planning for the possibility of supporting multiple languages. 
The solution will allow prospective learners that prefer a structured approach without personal tutoring to access files from the teacher, to facilitate their own learning.
\\\\
Based on these considerations we define the following problem statement:
\begin{quote}
    \textit{How can we design and implement a scalable, web-based solution to connect prospective learners and personal tutors, and allow tutors to publish material to accommodate different learning styles?}
\end{quote}
