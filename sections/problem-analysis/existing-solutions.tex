\section{Existing solutions}
There are many existing solutions for getting into contact with tutors, personal teachers, and places to seek information to learn new skills.
Most of these existing solutions are similar to \textit{Mentor Denmark}.  
In this section, we will only describe a few of them, mainly those with significant differences, as well as \textit{Mentor Denmark}.

\subsubsection{Mentor Denmark}
Mentor Denmark is the largest Scandinavian company for purchasing private teaching \cite{skandinaviens-stoeste-lektiefirma}.
There are multiple companies with a similar business model, where it offers connecting personal teachers and students. 
The common element to many of these companies is that the user contacts the company, and will be assigned a teacher.
Mentor Denmark employ the teachers and are as a result of this responsible for their quality. 
They screen the applicants when receiving applications for the position as a teacher, but also educate them. 
Most people would expect a certain level of quality because of the requirements for being a teacher.
Most of these companies focus on helping with homework in primary school or high school.

\subsubsection{Menti}
Unlike Mentor Denmark, the user contacts the tutor directly instead of contacting the company.
Menti also allows tutors to teach in other subjects than the traditional ones in public schools and high schools, such as teaching swimming or specific programming languages.
If the user wants to contact a tutor, they have to fill out an email form, and then the tutor will contact them.
Typically, the teaching will either be in person or on Skype.
The only requirement to register as a teacher on Menti is to fill out a form followed by an email confirmation.
When everyone can become a teacher in this way, the quality of the teachers cannot be assured.

\subsubsection{Udemy}
Udemy is a prevalent learning tool with around 100.000 online courses \cite{udemy}.
Unlike Menti and Mentor Denmark, Udemy is not for personal teaching. 
The user purchases access to a course with videos and possibly exercises and quizzes. 
The user can see statistics, such as how many people enrolled have for the course, how many rated the course and the average rating for the course.

\subsubsection{Superprof}
The design of \textit{Superprof} has many similarities to Menti.
Superprof is, however, a much bigger company as they advertise on their front page that they have a community of 8.7 million personal tutors \cite{superprof} and their services are available in 27 countries.
Superprof is not available in Denmark at this time.

\subsubsection{Comparison}
On \autoref{table:compare-companies} we compare some of the differences that the different companies have.
Mentor Danmark sets the price for services, so the tutors do not have a say in it.
For Menti and Superprof, the tutors can set their prices on the website. 
The prices in the table for these two companies show where the prices often are between where the low number is typically the lowest, and the highest number is costly.
For \textit{Udemy} a course is often priced around \$ 12 or less. 
There are, however, many more expensive courses on \textit{Udemy}.

\begin{table}[h]
    \begin{tabular}{|p{2cm}|p{2cm}|p{2cm}|p{2cm}|p{2cm}|p{2cm}|}
    \hline
    Company name:  & Possible to contact tutor directly: & Possible to upload material & Possibility for personal tutoring & Price for service                               & Available in Denmark \\ \hline
    Mentor Danmark & No                                  & No                          & Yes                               & 249 - 329 dkk an hour                           & Yes                  \\ \hline
    Menti          & Yes                                 & No                          & Yes                               & 120 - 1000 dkk an hour                          & Yes                  \\ \hline
    Udemy          & Yes                                 & Yes                         & No                                & Most often around \$ 12 or less for each course & Yes                  \\ \hline
    Superprof      & Yes                                 & No                          & Yes                               & 120 - 1000 dkk an hour                          & No                   \\ \hline
    \end{tabular}
    \caption{This table compares features for the previous mentioned companies.}
    \label{table:compare-companies}
\end{table}
