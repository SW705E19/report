\section{Existing solutions}
There are many existing solutions for getting into contact with tutors and personal teachers.
Most of these existing solutions are similar to \textit{Mentor Denmark}.  
In this section we will only describe a few of them, mainly those with the big differences, as well as \textit{Mentor Denmark}.

\subsubsection{Mentor Denmark}
Mentor Denmark is the largest Scandinavian company for purchasing private teaching \cite{skandinaviens-stoeste-lektiefirma}.
There are multiple companies with a similar business model, where you unite a personal teacher and a student. 
Common to many of these companies is that you contact the company, and then they will suggest a teacher for you.
Mentor Denmark employ the teachers, and are hereby responsible for their quality. 
They screen the applicants when receiving applications for the position as teacher, but also educate them. 
Most people would expect a certain level of quality because of the requirements for being a teacher.
Most of these companies focus on helping with homework in primary school or high school.

\subsubsection{Menti}
Unlike \textit{Mentor Denmark}, you contact the tutor directly instead of contacting the company to assign you a tutor.
Menti also allows tutors to teach in other subjects than the traditional ones in public school and high school, such as teaching swimming or specific programming languages.
If you want to contact a tutor, you have to fill out an email form and then the tutor will contact you.
Typically the teaching will either be in person or on Skype.
To become a teacher on \textit{Menti}, all you have to do is fill out a form and confirm the email.
When everyone can become a teacher, a lower level of quality is to be expected.

\subsubsection{Udemy}
Udemy is a very popular learning tool with around 100.000 online courses \cite{udemy}.
Unlike \textit{Menti} and \textit{Mentor Denmark}, this is not for personal teaching. 
You purchase access to a course with videos and possibly exercises and quizzes. 
You are able to see statistics such as how many people enrolled for the course, how many rated the course and the average rating for the course.

\subsubsection{Superprof}
The design of \textit{Superprof} has many similarities to \textit{Menti}.
This is however a much bigger company as they advertise on their frontpage that they have a community of 8.7 million personal tutors \cite{superprof}.