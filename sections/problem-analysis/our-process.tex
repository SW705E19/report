\section{Our process}\label{sec:our-process}
The process for our group was inspired by scrum, from which we used the following principles:

\begin{itemize}
    \item Sprints 
    \item Sprint planning 
    \item Planning poker 
    \item Product owner 
    \item Sprint backlog 
    \item Product backlog   
    \item Product increment 
    \item Retrospectives 
\end{itemize}

\noindent
We chose to have 1 week sprints, as they allowed us to continually keep up to date with what each other was doing, as well as refine the tasks on a weekly basis.
We also experienced from last semester that shorter sprints were more productive as we had to communicate more and the tasks had to be smaller and more well defined, leading to a lack of confusion as to what the different tasks would entail.
Each week there would be a person assigned as the product owner role, such that it rotated between all members.
The role of the product owner was to prepare the next week's sprint and define a DoD (definition of done) on all tasks he chose to bring from the product backlog to the upcoming sprint backlog.
Along with this the product owner managed our contact with the supervisor and had the responsibility of booking rooms to work or have meetings in.
Each Thursday we conducted a sprint planning where planning poker was used to estimate the workload of the tasks.
During the planning poker we also discussed the importance of the different tasks and which parts of a task were deemed insignificant and what should be focused on.
Each task in the sprint backlog would have a DoD and these would be discussed so that everyone had an agreement on how the task should be completed.
At the end of each sprint a product increment was created. 
Before the next sprint planning occurred a sprint retrospective was made to discuss the workflow of the previous sprint.
During the sprint retrospectives we discussed issues that were had during the sprint, and potential changes to the process.
