\section{Using personas to help design}\label{sec:personas}
In addition to having a scalable system, it should also be easy to use for the users.
To assist in constructing a usable design, it should be visualized, such that ideas can be clarified and evaluated through these visualizations \cite{DEB}.
One way to perform such a visualization is through personas.
Personas are fictive profiles of the possible users of the system the designer is designing for.
They are an amalgamation of details about a person, such as a name and a background, as well as goals and aspirations the person might have \cite{DEB}.
Personas want to achieve certain goals and undertake certain activities through the use of the system being designed and exist to remind designers that they do not design for themselves.
They are a method of allowing designers to think about the system from a different perspective.
Since different types of people will use the system, it is necessary to develop several personas, to accurately capture a wide variety of goals and aspirations.
\\\\
The following list defines the possible personas this system would be developed for:
\begin{itemize}
    \item Parent looking for a tutoring service for their child
    \item Tutor looking to make their services available through the system with varying technological experience
    \item Persons of varying age that want to learn something new through personal tutoring
\end{itemize}
\noindent
\textbf{Peter}
\\
Peter is 45 years old.
He works as a carpenter, and rarely uses computers.
He gets up at 6.00 AM, eats breakfast, and arrives at work at 7.00 AM.
He usually gets home around 3.40 PM, but sometimes has to stay to work a bit longer to finish a part of a project.
He is the father of Svend, who is 13 years old.
Peter is recently divorced, and Svend is having trouble coping with the new lifestyle.
Svend, who used to be an active boy, has started neglecting physical activity and is both unhealthy and unhappy.
Peter wants to help Svend find joy in being active again. 
\\\\
\textbf{Helle}
\\
Helle is 21 years old.
She has recently dropped out of college and does not have a job.
While searching for a job, she grows stagnant and longs for a change to her monotonous day.
To feel a sense of accomplishment, she decides that she wants to learn something new that could give her a continuous sense of improvement.
She regularly uses computers, and has decided to search for a teacher on the internet, but does not have a specific subject in mind.
She wants to find a teacher that can tutor her in person.
\\\\
\textbf{Anton}
\\
Anton is 33 years old.
He works as a teacher for a small school, teaching Danish and physical education for children between the ages of 8-10.
Since he was ten years old, he has been passionate about swimming.
Throughout years of consistent training, he has advanced to a level that is highly advanced, but not entirely professional.
He enjoys talking about swimming with others.
He wants to pass his passion on to others, combining it with his knowledge of teaching.
He is familiar with computers and has decided to offer his services on an internet system.
\\\\
\textbf{Jensine}
\\
Jensine is 54 years old.
She grew up in a strict household and had many chores.
She also regularly spent time with her mother learning to sew.
Jensine loves sewing and wants to ensure the next generation learns this valuable skill.
After the death of her mother, she has decided to take action and wants to teach sewing the same way her mother did.
She never got used to technology, and computers are a mostly foreign concept for her.
However, she decides that making her services available on an internet system would be the best course of action.
\\\\
\textbf{Storm}
\\
Storm is 13 years old.
He is a quiet boy who gets bullied at school. 
He struggles to get through the day, as no particular subject seems to catch his interest, and he shows no particular aptitude for any specific subject.
He finds solace in playing video games, especially one particular game. 
He wants to improve at this game.
He lives with his parents in a fairly remote location, and getting a tutor to show up at this location physically is not feasible.
As such, he would rather find a tutor that can assist through the internet.
