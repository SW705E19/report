\section{Summary of the problem analysis}
In the preceding chapter we examined some existing solutions related to the problem of making accessible teaching services.
We looked at \textit{Mentor Danmark}, \textit{Menti}, \textit{Udemy}, and \textit{Superprof}.
This revealed some differences in the ways tutors set their prices.
Mentor Danmark sets the prices for their services, while Menti and Superprof lets the tutors set the prices.
For solutions that let tutors upload materials, users were unable to receive personal tutoring, and vice versa.
We defined a set of personas, which are fictive profiles of possible users, to help when constructing the design.
We determined some different types of scalability, and found some important principles to follow when making a system with scalability in mind.
These are:
\begin{itemize}
    \item Scalability in performance, availability, maintenance and expenditure
    \item Design for at least two of everything
    \item Scale horizontally
    \item Build API first
    \item Cache data
    \item Rely on eventual consistency
    \item Make maintenance and automation easy
    \item Write asynchronous code
    \item Do not store information about states in the system servers
\end{itemize}
Finally we defined our process, basing it around short, weekly sprints that included planning with planning poker, backlogs for tasks, the product owner role, product increments and retrospectives.
