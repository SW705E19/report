\section{Problem statement}
In \autoref{chap:introduction} we introduced an initial problem statement stating the following:
\begin{quote}
    \textit{How can we design and implement a scalable, web-based solution to connect prospective learners and personal tutors, and allow tutors to publish material to accommodate different learning styles?}
\end{quote}
As the project progressed and we defined the MoSCoW model in \autoref{sec:moscow}, the goals of the project changed and we came to realize that this problem statement did not fit what the goal of the project ended up being.
Based on revisions of what the minimum viable project should be, we decided to remove the requirement of tutors being able to upload material as a must have element of the system.
This reflected poorly on the initial problem statement, as it mentions accommodating different learning styles by allowing tutors to upload material.
The problem pivoted to focus more on users simply using the system, and having a recommender system to predict which services a user might like.
As such, we revised the initial problem statement to the final problem statement:
\begin{quote}
    \textit{How can we design and implement a scalable, web-based solution to connect prospective learners and personal tutors, and highlight services that a user is expected to like?}
\end{quote}
